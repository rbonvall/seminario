\section*{Miércoles 20 de julio}

\((X, \norm{\pdot})\) espacio normado.
Se dice que es un \emph{espacio de Banach} ssi
toda sucesión de Cauchy en \(X\) converge en \(X\).

\(\{x_n\}_{n ∈ \N}\) sucesión en \(X\), se dice que es de Cauchy ssi:
\begin{equation}
  ∀ ε > 0\quad
  ∃ N ∈ \N\quad
  ∀ p, q ∈ \N\text{ con }p, q ≥ N:\quad
  \norm{x_p - x_q} < ε.
\end{equation}

Ejemplo.
Consideremos en \((\Q, \abs{\pdot})\)
la sucesión:
\begin{equation}
  x_n = \text{«\(n\) primeros dígitos de \(π\)»},
\end{equation}
i.e.:
\begin{equation}
  \{x_n\}_{n ∈ \N} = \{3,\;3.1,\;3.14,\;3.141,\;3.1415,\;3.14159,\;\ldots\}.
\end{equation}
\(\{x_n\}_{n ∈ \N}\) es sucesión de Cauchy en \(\Q\),
pero no converge en \(\Q\),
pues \(\lim_{n → ∞} x_n = π \not∈ \Q\).
Se dice que \(\Q\) no es completo.
(Tarea: demostrar que \(π\) es irracional).

Consideremos la serie armónica \(s_n = \sum_{k = 1}^n \frac{1}{k}\).
Claramente, \(s_n ∈ \Q\).
No obstante, consideremos esta sucesión en \(\R\).
\begin{equation}
  s_n - s_m = \sum_{k = n + 1}^{m} =
  \frac{1}{n + 1} +
  \frac{1}{n + 2} +
  \cdots +
  \frac{1}{m}.
\end{equation}
Esta diferencia puede ser tan grande como se quiera,
eligiendo \(m\) suficientemente grande.
\(\{s_n\}\) no es una sucesión de Cauchy.

Tarea: demostrar que:
\begin{equation}
  \left\{
    \sum_{k = 1}^n \frac{1}{k} - \log n
  \right\}_{n ∈ \N}
\end{equation}
converge.
El límite es la constante de Euler-Mascheroni \(γ\).

Tarea: demostrar que la sucesión\(\{\sin n: n ∈ \N\} ⊆ [-1, 1]\)
(que es divergente) es densa en \([-1, 1]\).

En lugar del valor absoluto,
consideremos como norma
la distancia entre los puntos sobre la circunferencia
de la proyección estereográfica.
Con esta norma, \(\{s_n\}\) sí converge.

(Falta poner diagrama).

\(X ≠ ∅\) espacio topológico compacto.

En los compactos vale el \emph{teorema de intersección de Cantor}.
\(\mathcal{Z} = \{A: A\text{ cerrado}\}\) centrado,
i.e. la intersección de cualquier subsistema finito de \(\mathcal{Z}\) es no vacía:
\begin{equation}
  \bigcap_{k = 1}^{n} A_k ≠ ∅\quad
  ∀\text{ subsistema } \{A_k\}_{k = 1}^{n ∈ \N} ⊂ \Z,
\end{equation}
Entonces, \(\bigcap_{A ∈ \mathcal{Z}} A ≠ ∅\).

Ejemplos de compactos:
\begin{itemize}
    \item el intervalo \([0, 1] ⊂ \R\),
    \item el disco unitario \(\mathbb{D} ⊂ \C\) (incluyendo su borde),
    \item la circunferencia unitaria \(S = ∂\mathbb{D} ⊂ \C\).
\end{itemize}

\(X ≠ ∅\) espacio topológico compacto.
\begin{equation}
  C(X, \R)\defeq
  \{f\colon X\to\R \mid f\text{ es continua}\}.
\end{equation}
Recordemos que «\(f\) es continua» significa que:
\begin{equation}
  ∀V ⊆ \R\text{ abierto}:
  f^{-1}(V) ⊂ X\text{ es abierto en }X.
\end{equation}

Definamos una estructura de espacio vectorial en \(\R\) sobre \(C(X, \R)\):
\begin{equation}
  \label{eq:comb-lineal-cxr}
  \bigl(αf + βg\bigr)(x)\defeq
  \underbrace{αf(x) + βg(x)}_{∈ \R}\quad
  ∀x ∈ X,\; α, β ∈ \R,\; f, g ∈ C(X, \R).
\end{equation}
\eqref{eq:comb-lineal-cxr} define una función \(αf + βg\colon X\to\R\).
Hay que verificar que ésta es una función continua, i.e., que:
\begin{equation}
  \bigl(αf + βg\bigr)^{-1}(]a, b[)\defeq
  \{x ∈ X: αf(x) + βg(x) ∈ ]a, b[\}
\end{equation}
es un abierto en \(X\) (¡tarea!).

Definamos ahora una norma sobre este espacio vectorial:
\begin{align}
  \norm{\pdot}\colon C(X, \R) &\to\R \\
  f &\mapsto\norm{f}_{*}\defeq
    \sup_{x ∈ X}\abs{f(x)}
    \stackrel{†}{=}
    \max_{x ∈ X}\abs{f(x)}.
\end{align}
La igualdad \(†\) es consecuencia del teorema de Weierstraß:
una función continua en un compacto alcanza su máximo.
Hay que verificar que \(\norm{\pdot}\) efectivamente es una norma (¡tarea!).

\(\bigl(C(X, \R), \norm{\pdot}\bigr)\) espacio vectorial normado.
Por demostrar: éste es un espacio de Banach.
