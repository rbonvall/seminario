\section*{Miércoles 20 de julio}

\((X, \norm{\pdot})\) espacio normado.
Se dice que es un \emph{espacio de Banach} ssi
toda sucesión de Cauchy en \(X\) converge en \(X\).

\(\{x_n\}_{n ∈ \N}\) sucesión en \(X\), se dice que es de Cauchy ssi:
\begin{equation}
  ∀ ε > 0\quad
  ∃ N ∈ \N\quad
  ∀ p, q ∈ \N\text{ con }p, q ≥ N:\quad
  \norm{x_p - x_q} < ε.
\end{equation}

Ejemplo.
Consideremos en \((\Q, \abs{\pdot})\)
la sucesión:
\begin{equation}
  x_n = \text{«\(n\) primeros dígitos de \(π\)»},
\end{equation}
i.e.:
\begin{equation}
  \{x_n\}_{n ∈ \N} = \{3,\;3.1,\;3.14,\;3.141,\;3.1415,\;3.14159,\;\ldots\}.
\end{equation}
\(\{x_n\}_{n ∈ \N}\) es sucesión de Cauchy en \(\Q\),
pero no converge en \(\Q\),
pues \(\lim_{n → ∞} x_n = π \not∈ \Q\).
Se dice que \(\Q\) no es completo.
(Tarea: demostrar que \(π\) es irracional).

Consideremos la serie armónica \(s_n = \sum_{k = 1}^n \frac{1}{k}\).
Claramente, \(s_n ∈ \Q\).
No obstante, consideremos esta sucesión en \(\R\).
\begin{equation}
  s_n - s_m = \sum_{k = n + 1}^{m} =
  \frac{1}{n + 1} +
  \frac{1}{n + 2} +
  \cdots +
  \frac{1}{m}.
\end{equation}
Esta diferencia puede ser tan grande como se quiera,
eligiendo \(m\) suficientemente grande.
\(\{s_n\}\) no es una sucesión de Cauchy.

Tarea: demostrar que:
\begin{equation}
  \left\{
    \sum_{k = 1}^n \frac{1}{k} - \log n
  \right\}_{n ∈ \N}
\end{equation}
converge.
El límite es la constante de Euler-Mascheroni \(γ\).

Tarea: demostrar que la sucesión\(\{\sin n: n ∈ \N\} ⊆ [-1, 1]\)
(que es divergente) es densa en \([-1, 1]\).

En lugar del valor absoluto,
consideremos como norma
la distancia entre los puntos sobre la circunferencia
de la proyección estereográfica.
Con esta norma, \(\{s_n\}\) sí converge.

(Falta poner diagrama).

Con esta norma, \(\{s_n\}\) sí converge.
