\section*{Miércoles 6 de julio}

\(X\), \(Y\) espacios normados.
\(T\colon X\to Y\) lineal y continua.
\begin{equation}
  \norm{T}\defeq\inf\bigl\{
    C ≥ 0: \norm{T(x)} < C\norm{x}\quad ∀x ∈ X
  \bigr\}
\end{equation}
Demostremos que:
\begin{equation}
  \sup_{0 ≠ x ∈ X} \frac{\norm{T(x)}}{\norm{x}} ≤ \norm{T}.
\end{equation}

\begin{align}
  ∀C ∈ \{C ≥ 0: \ldots\}:\quad& \norm{T(x)} ≤ C\norm{x}\quad ∀x ∈ X \\
    ⇒\quad& \frac{\norm{T(x)}}{\norm{x}} ≤ C \quad ∀ 0 ≠ x ∈ X \\
    ⇒\quad& ∀C ∈ \{\ldots\}\quad ∀0 ≠ x ∈ X\quad
            \underbrace{\frac{\norm{T(x)}}{\norm{x}}}_{\text{aplicar sup}}
            ≤
            \underbrace{C}_{\text{aplicar ínf}} \\
    ⇒\quad& \sup_{0 ≠ x ∈ X} \frac{\norm{T(x)}}{\norm{x}} ≤
            \inf_{C ∈ \{\ldots\}} C = \norm{T}
\end{align}

Supongamos que
\(\displaystyle\sup_{0 ≠ x ∈ X} \frac{\norm{T(x)}}{\norm{x}} =: D < \norm{T}\).
Por ejemplo: \(D = \norm{T - ε}\).
\begin{align}
  ⇒\quad& \norm{T(x)} ≤ D\norm{x}\quad ∀x ∈ X \\
  ⇒\quad& D ∈ \{C > 0: \ldots\} \\
  ⇒\quad& \norm{T}\defeq \inf\{C > 0: \ldots\} ≤ D < \norm{T}\quad\Rightarrow\!\Leftarrow.
\end{align}

Conclusión: \(\norm{T}\defeq\inf\{C > 0: \ldots\} = \sup\frac{\norm{T(x)}}{\norm{x}}\).

Tarea: verificar que es una norma.

\(X\) espacio normado.
Una \emph{sucesión de Cauchy \(x_n\) en \(X\)} es una sucesión tal que:
\begin{equation}
  ∀ε > 0\quad
  ∃N = N(ε) ∈\N\quad
  ∀p, q ∈\N \text{ con } p, q > N:\quad
  \norm{x_p - x_q} < ε.
\end{equation}

Definición:
\((X. \norm{\;\cdot\;}\) se dice que es un \emph{espacio de Banach}
ssi toda sucesión de Cauchy \(\{x_n\}\) en \(X\) converge en \(X\), i.e.,
para toda sucesión de Cauchy \(x_n\) en \(X\):
\begin{equation}
  ∃ x ∈ X:\quad
  x_n\xrightarrow{\norm{\;\cdot\;}} x
  \text{ para }n → ∞.
\end{equation}

Próxima clase: \(C(X, \C)\) es un espacio de Banach.
