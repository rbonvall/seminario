\section*{Viernes 10 de junio---La topología cuociente}

\((X, \T)\) espacio topológico, \(X ⊆ \powerset(X)\).

\(\rel ⊆ X\times X\) relación de equivalencia en \(X\).

Por ejemplo,
\((\R, \T_C)\), donde \(\T_C\) es la topología generada
por los intervalos:
\begin{equation}
  \T_C = \{
    ]a, b[ ⊆ \R, a < b,\; a, b ∈ \R
  \}
\end{equation}
y \(\rel\) es la siguiente relación de equivalencia
(es refleja, simétrica y transitiva):
\begin{equation}
  \rel:\quad x\rel y \stackrel{\text{Def}}{⇔} x - y ∈ \Q.
\end{equation}

¿\(π\rel e\)? Esto es, ¿\(π - e ∈ \Q\)?
(¡Demostrar que un número es irracional es muy complicado!).

Sabemos que \(π\rel π ± 1, π ± \frac{191}{273},\) etc.

Podemos obtener la clase de equivalencia de \(π\) en la relación \(\rel\):
\begin{equation}
  [π]_{\rel} = \{ π + \frac{p}{q}: \quad p, q ∈ \Z,\quad q > 0 \}.
\end{equation}
\begin{equation}
  (π, π + {\textstyle \frac{p}{q}}) ∈ \rel ⊂ \R\times\R,\quad p, q ∈ \Z,\quad q > 0.
\end{equation}
\(\R\) queda particionado en clases de equivalencia:
\begin{align}
  π\colon \R &\to \R/\rel, \\
  x &\mapsto [x]_{\rel}.
\end{align}
\(\pi\) (¡no el número!) se llama \emph{proyección}.

En general:
\begin{align}
  π\colon X &\to X/\rel, \\
  x &\mapsto [x]_{\rel}.
\end{align}
Tenemos una topología \(\T\) en el espacio \(X\),
y queremos definir una topología adecuada en el espacio \(X/\rel\).

\begin{equation}
  \mathscr{F}\defeq\{ V ⊆ X/\rel:\quad π^{-1}(V) ∈ \T \}
\end{equation}
¿Es \(\mathscr{F}\) una topología?
\begin{enumerate}
    \item
      ¿\(∅ ∈ \mathscr{F}\)?
      Tomemos \(V = ∅\).
      \begin{equation}
        π^{-1}(∅) = \{x ∈ X: π(x) ∈ ∅\} = ∅ ∈ \T ⇒ ∅ ∈ \mathscr{F}.
      \end{equation}

      ¿\(X/\rel ∈ \mathscr{F}\)?
      Tomemos \(V = X/\rel\).
      \begin{equation}
        π^{-1}(X/\rel) = \{x ∈ X: π(x) ∈ X/\rel\} = X ∈ \T ⇒ X/\rel ∈ \mathscr{F}.
      \end{equation}
    \item
      \(V_α ⊆ X/\rel, α ∈ \mathscr{A}\).
      ¿\(\bigcup_{α ∈ \mathscr{A}} V_α ∈ \mathscr{F}\)?

      (Demostración por transcribir).

    \item
      ¿\(\bigcap_{k = 1}^{n} V_k ∈ \mathscr{F}\)?
      Tarea.
\end{enumerate}

Yapa. \(π\colon (X, \T)\to (X/\rel, \mathscr{F})\) es automáticamente continua,
ya que está construída de modo que las preimágenes de abiertos
son abiertos (\(∈ \mathscr{F}\)).
Más aún, \(\mathscr{F}\) es la topología más gruesa de \(X/\rel\)
que hace continua la proyección.
