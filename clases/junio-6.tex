\section*{Lunes 6 de junio---El grupo fundamental (por Rafael Plaza)}

\(G\) conjunto. \(*\colon G × G\to G\).
Se dice que \(*\) dota a \(G\) con una \emph{estructura de grupo}
si se cumplen las siguientes condiciones:
\begin{enumerate}
  \item asociatividad: \(a * (b * c) = (a * b) * c\),
  \item existencia de la identidad \(e\):
    \(a * e = e * a = a\),
  \item existencia del inverso:
    \(a * a^{-1} = e\).
\end{enumerate}
Ejemplo: \((\Z, +)\).

Sean \(X, Y\) dos espacios topológicos.
Una función \(f\colon X\to Y\) es un \emph{homeomorfismo}
si cumple las siguientes propiedades:
\begin{enumerate}
  \item \(f\) es biyectiva,
  \item \(f\) es continua,
  \item \(f^{-1}\) es continua.
\end{enumerate}
\(X\) e \(Y\) son \emph{homeomorfos}
si existe  un homeomorfismo entre ellos.

Por ejemplo,
\(X = ]{-\frac{π}{2}}, {+\frac{π}{2}}[\) e
\(Y = \R\) son homeomorfos,
pues \(f(x) = \tan x\) es un homeomorfismo.

Otro ejemplo: \(X = [0, 2π[\) e \(Y = S^1\).
¿Es \((\cos x, \sin x)\) un homeomorfismo?
No, porque la inversa no es continua:

(falta diagrama).

¿Cómo se demuestra que no existe ningún homeomorfismo?

(Falta la segunda parte).
