\section*{Viernes 22 de julio}

En \(\R^n\) o \(\C^n\) o espacios de dimensión finita,
los conjuntos compactos son exactamente los conjuntos cerrados y acotados.

Por ejemplo,
en \(\R\) los intervalos cerrados \([a, b]\) son compactos.
En \(\C\), los discos cerrados:
\begin{equation}
  \bar{D}(z_0, ρ) = \{
    z ∈ \C:
    \abs{z - z_0} ≤ ρ
  \},\quad
  ρ ≥ 0.
\end{equation}
En \(\R^n\), las bolas cerradas:
\begin{equation}
  \bar{B}(x_0, ρ) = \{
    x ∈ \R^n:
    \norm{x - x_0}_2 ≤ ρ
  \}.
\end{equation}

Colocando estructura de espacio vectorial:
\begin{equation}
  α, β ∈ \C,\quad
  f, g ∈ C(X, \C)\quad
  \rightsquigarrow\quad
  αf + βg.\quad
\end{equation}
Dado un abierto \(V ⊆ \C\),
hay que probar que:
\begin{equation}
  \bigl(αf + βg\bigr)^{-1}(V) ∈ \T,
\end{equation}
i.e., que \(\bigl(αf + βg\bigr)^{-1}(V)\) es abierto.

(Falta el diagrama).

Colocando estructura de norma sobre \(C(X, \C)\):
\begin{align}
  \norm{\pdot}\colon C(X, \C) &\to\R \\
  f &\mapsto\norm{f}\defeq
    \sup_{x ∈ X}\abs{f(x)}.
\end{align}
Teorema de Weierstraß:
en compactos se cumple que \(
  \sup_{x ∈ X}\abs{f(x)} =
  \max_{x ∈ X}\abs{f(x)}
\).


