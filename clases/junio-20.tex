\section*{Lunes 20 de junio}

\((X, \norm{\pdot}_X)\), \(Y ⊆ X\) s.espacio cerrado.
\(\mathfrak{Y}: u\mathbin{\mathfrak{Y}} v ⇒ u - v ∈ Y\)
es una relación de equivalencia.

\begin{equation}
  X/Y\defeq X/\mathfrak{Y} = \bigl\{[u]_{\mathfrak{Y}}: u ∈ X\}
\end{equation}

\begin{align}
  π\colon X &\to X/Y,\text{ proyección canónica} \\
  u &\mapsto [u]_Y
\end{align}
\(V\text{ abierto en }X/Y ⇔ π^{-1}(V)\text{ abierto en }X/Y\).

Resulta que la topología inducida por \(π\) sobre \(X/Y\)
es una topología nórmica:
\begin{equation}
  \norm{[X]_Y}_{X/Y} = \inf_{y ∈ Y} \norm{x + y}_X.
\end{equation}

Ya vimos la noción de continuidad en los espacios normados.
\(f\colon X\to Y\) es continua en \(x_0 ∈ X\) ssi:
\begin{equation}
  ∀ ε > 0\quad
  ∃ δ > 0\quad
  ∀ u ∈ X\text{ con }\norm{u - x_0}_X < δ(x):\quad
  \norm{f(u) - f(x_0)}_Y < ε,
\end{equation}
i.e. \(\lim_{u → 0} f(u) = f(x_0)\).

Se dice que \(f\colon X\to Y\) es cerrada ssi:
\begin{equation}
  \graph(f)\defeq\{(x, y):\quad x ∈ X, y = f(x)\}
\end{equation}
es cerrado en \(X × Y\),
donde \(X × Y\) posee en forma natural una topología nórmica.

\(X × Y\) tiene una estructura de espacio vectorial:
\begin{equation}
  α (u, v) + β (x, y) \defeq (αu + βx, αv + βy),
\end{equation}
donde \(α, β ∈ \K,\, u, x ∈ X,\, v, y ∈ Y\). Además:
\begin{equation}
  \norm{(x, y)}_{X × Y} =
  \Bigl\lVert
    \underbrace{
      \bigl(
        \norm{x}_X, \norm{y}_Y
      \bigr)
    }_{∈ \R^2}
  \Bigr\rVert_{\R^2 \text{ (cualquier norma)}}
\end{equation}

\(f\colon X\to Y\) se dice \emph{Lipschitz-continua} en \(X\) ssi:
\begin{equation}
  ∃ M > 0\quad
  ∀ u, v ∈ X:\quad
  \norm{f(u) - f(v)}_Y ≤ M\norm{u - v}_X.
\end{equation}
El ínfimo de todas estas constantes se denota por \(L\),
y se llama \emph{constante de Lipschitz} de \(f\).

\((X, \norm{\pdot})\) espacio normado, \(U ⊆ X\) usualmente compacto.
\begin{align}
  C(U, Y) &\defeq \{
    f\colon U\to Y:\;
    \text{\(f\) es continua sobre \(U\)}
  \}, \\
  BC(\bar{U}, Y) &\defeq \{
    f\colon\bar{U}\to Y:\;
    \text{\(f\) es continua sobre \(U\) y \(f\) es acotada en \(\bar{U}\)}
  \}.
\end{align}
\(f\colon\bar{U}\to Y\) es uniformemente continua sobre \(\bar{U}\) ssi:
\begin{equation}
  ∀ ε > 0\quad
  ∃ δ > 0\quad
  ∀ x ∈ \bar{U}\quad
  ∀ u ∈ \bar{U}\text{ con }\norm{u - x}_X < δ
\end{equation}
se tiene que \(\norm{f(u) - f(x)}_Y < ε\).
Es decir, ahora \(δ\) no depende de \(x\).

Problema: ¿las propiedades de continuidad y acotadura implican continuidad uniforme?
¿Le ayuda si \(\bar{U}\) es compacto? (Ver Diodoné).
