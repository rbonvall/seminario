\section*{Viernes 1 de julio---Tarea con nota}

\(X = \Q\) considerado como espacio vectorial sobre \(\Q\), i.e.:
\begin{equation}
  x, y ∈ \Q,\quad
  α, β ∈ \Q\quad
  →\quad
  αx + βy ∈ \Q.
\end{equation}

Sea \(p ∈\N\) primo, \(p ≥ 2\).
Entonces, todo \(x ∈ \Q\) se puede escribir como:
\begin{equation}
  x = p^k \frac{s}{t},\quad
  s, t ∈ \Z,\quad
  \gcd(s, t) = 1,\quad
  p\nmid s,\quad
  p\nmid t.
\end{equation}
(Todo esto significa que \(p^k \frac{s}{t}\)
no se puede simplificar por ningún lado).

Consideremos la «valuación» \(\abs{\pdot}_p\):
\begin{align}
  \abs{\pdot}_p \colon \Q &\to \Q \\
  x &\mapsto \abs{x}_p = p^{-k}, \\
  0 &\mapsto \abs{0}_p = 0.
\end{align}

Tarea. Verificar o refutar lo siguiente:
\begin{enumerate}
  \item \(\abs{x}_p ≥ 0\).
  \item \(\abs{x}_p = 0 ⇔ x = 0\).
  \item \(λ ∈ \Q \text{ (escalar)}, x ∈ Q ⇒
          \abs{λx}_p = \abs{λ}_p \abs{x}_p.\)
  \item \(x, y ∈ \Q ⇒
          \abs{x + y}_p ≤ \abs{x}_p + \abs{y}_p\).
    \hfill (\(\mathord{≠}\bigtriangleup\)'r)
  \item \(x, y ∈ \Q ⇒
          \abs{x + y}_p ≤ \sup\bigl\{\abs{x}_p, \abs{y}_p\bigr\}\).
    \hfill (\(\mathord{≠}\bigtriangleup\)'r fuerte)
  \item
    Recordemos que una sucesión de Cauchy \(\{x_n\}_{n ∈ \N}\) en un espacio \(X\)
    es una sucesión tal que:
    \begin{equation}
      ∀ε > 0\quad
      ∃N ∈ \N\quad
      ∀p, q ∈ \N\text{ con } p, q ≥ N:\quad
      \norm{x_p - x_q} < ε.
    \end{equation}
    (Relajadamente: los términos se van «amontonando»).

    Los números reales se definen
    como los límites de las sucesiones de Cauchy en \(\Q\):
    \begin{equation}
      \R\sim\frac{
        \text{conjunto de todas las sucesiones de Cauchy c/r al valor absoluto en \(\Q\)}
      }{\rel},
    \end{equation}
    donde:
    \begin{equation}
      \rel\colon\quad
      \{x_n\}\rel\{y_n\} \stackrel{\text{def}}{⇔}
      \{x_n - y_n\}\text{ es una sucesión nula},
    \end{equation}
    i.e. \(x_n - y_n → 0 \).

    Este proceso se llama \emph{completación de los racionales}
    con respecto al valor absoluto.
    Pero ahora tenemos las valuaciones \(\abs{\pdot}_p\),
    que son otras formas de valores absolutos,
    y podemos considerar sucesiones de Cauchy en \(\Q\) con respecto a ellas,
    y completar \(\Q\) con respecto a ellas
    para obtener la completación \(p\)-ádica de \(\Q\),
    denotada por \(\Q_p\).

    Demostrar o refutar: \(1 < p, q ∈ \N\text{ primos} ⇒ Q_p ≠ Q_q\).
    (Ver Diodoné).
\end{enumerate}

Las operaciones de suma de vectores y multiplicación de vectores por escalares
son continuas:
%
\begin{align}
  +\colon X × X &\to X    & \cdot\colon \K × X &\to X \\
  (u, v) &\mapsto u + v   & (λ, x) &\mapsto λx
\end{align}
\begin{equation}
  \left.
  \begin{split}
    u_n\xrightarrow{\norm{\pdot}} u &\text{ en \(X\) para } n → ∞ \\
    v_n\xrightarrow{\norm{\pdot}} v &\text{ en \(X\) para } n → ∞ \\
  \end{split}
  \right\}
  \quad\Longrightarrow\quad
  u_n + v_n \xrightarrow{\norm{\pdot}} u + v
  \text{ para } n → ∞
\end{equation}
\begin{equation}
  \left.
  \begin{split}
    λ_n\xrightarrow{\abs{\pdot}} λ &\text{ en \(\K\) para } n → ∞ \\
    u_n\xrightarrow{\norm{\pdot}} u &\text{ en \(X\) para } n → ∞ \\
  \end{split}
  \right\}
  \quad\Longrightarrow\quad
  λ_n x_n \xrightarrow{\norm{\pdot}} λx
  \text{ para } n → ∞
\end{equation}

Todas estas propiedades se cumplen
cuando \((X, +, \cdot\K)\) es un espacio vectorial.

\(∅ ≠ K ⊆ X\) compacto,
i.e. todo recubrimiento abierto de K posee un recubrimiento finito.
\(f\colon K\to\C\) función continua sobre \(K\).
Entonces \(f\) es uniformemente  continua sobre \(K\).

