\section*{Viernes 27 de mayo}

(Falta transcribir una parte).

Objetivo del estudio de las topologías:
hacer más sólida la noción de continuidad.

Primer año: \(f\colon\R\to\R\) es continua ssi:
\begin{equation}
  ∀ε > 0\quad
  ∀x ∈ \R\quad
  ∃δ = δ(x, ε)\quad
  ∀y ∈ \R\text{ con }\abs{y - x} < δ:\quad
  \abs{f(y) - f(x)} < ε.
\end{equation}

Espacio métrico:
conjunto equipado con función \(σ\colon X\to\R\)
que satisface:
\begin{enumerate}
  \item \(σ(x, y) ≥ 0\quad ∀x, y ∈ X\),
  \item \(σ(x, z) ≤ σ(x, y) + σ(y, z)\quad ∀x, y, z ∈ X\),
  \item \(σ(x, y) = 0 ⇔ x = y\).
\end{enumerate}

\((X, \T_X), (Y, \T_Y)\) espacios topológicos.
\(f\colon X\to Y\) se dice continua ssi:
\begin{equation}
  ∀V ∈ \T_Y:\quad
  f^{-1}(V) ∈ \T_X.
\end{equation}
Ejemplo: en \(\R\):
\begin{equation}
  \mathfrak{M}\defeq
    \bigl\{ ] a, b[: a, b ∈ \R \bigr\} \cup
    \bigl\{ ] a, ∞[: a    ∈ \R \bigr\} \cup
    \bigl\{ ]-∞, b[: b    ∈ \R \bigr\}
\end{equation}
(\(\T_{\mathfrak{M}}\) es la topología de Borel).
