\section*{Viernes 27 de mayo}

\(X, \mathfrak{M} ∈ \powerset(X)\).
\begin{equation}
  \T_{\mathfrak{M}} =
  \bigcap
  \underbrace{
    \{\T \;|\; \T\text{ topología de \(X\) tal que }\mathfrak{M} ⊆ \T\}
  }_{\mathscr{S}}
\end{equation}

\begin{align}
  V_i ∈ \T ∈ \mathscr{S}\quad
  ∀i = 1, \ldots, n
    &\quad⇒\quad \bigcap_{i = 1}^{n} V_i ∈ \T\quad ∀\T ∈ \mathscr{S} \\
    &\quad⇒\quad \bigcap_{i = 1}^{n} V_i ∈ \bigcap\mathscr{S} \\
    &\quad⇒\quad \bigcap_{i = 1}^{n} V_i ∈ \T_\mathfrak{M}.
\end{align}

\(\T_\mathfrak{M}\) es la topología más gruesa que contiene a la familia \(\mathfrak{M}\).

Objetivo del estudio de las topologías:
hacer más sólida la noción de continuidad.

Primer año: \(f\colon\R\to\R\) es continua ssi:
\begin{equation}
  ∀ε > 0\quad
  ∀x ∈ \R\quad
  ∃δ = δ(x, ε)\quad
  ∀y ∈ \R\text{ con }\abs{y - x} < δ:\quad
  \abs{f(y) - f(x)} < ε.
\end{equation}

Espacio métrico:
conjunto equipado con función \(σ\colon X\to\R\)
que satisface:
\begin{enumerate}
  \item \(σ(x, y) ≥ 0\quad ∀x, y ∈ X\),
  \item \(σ(x, z) ≤ σ(x, y) + σ(y, z)\quad ∀x, y, z ∈ X\),
  \item \(σ(x, y) = 0 ⇔ x = y\).
\end{enumerate}

\((X, \T_X), (Y, \T_Y)\) espacios topológicos.
\(f\colon X\to Y\) se dice continua ssi:
\begin{equation}
  ∀V ∈ \T_Y:\quad
  f^{-1}(V) ∈ \T_X.
\end{equation}
Ejemplo: en \(\R\):
\begin{equation}
  \mathfrak{M}\defeq
    \bigl\{ ]{ a, b}[: a, b ∈ \R \bigr\} \;\cup\;
    \bigl\{ ]{ a, ∞}[: a    ∈ \R \bigr\} \;\cup\;
    \bigl\{ ]{-∞, b}[: b    ∈ \R \bigr\}
\end{equation}
(\(\T_{\mathfrak{M}}\) es la topología de Borel).

(Falta diagrama de función \(f\) de a continuación).

\(
  ]{-∞}, -\frac{1}{2}[ \;∪\;
  ]\frac{1}{2}, {+∞}[  \;∪\;
  \{0\}
\), ¿es abierto? No, por la presencia de \(\{0\}\).
Luego, \(f\) no es continua.
\(f\colon \R\setminus\{0\}\to\R\setminus\{0\}\) sí es continua.

\(X ≠ ∅\), \((Y, \T_Y)\) espacio topológico.
No tiene sentido decir si una función \(f\colon X\to Y\) es o no continua,
porque no hay topología en \(X\).

Problema: construir una topología en \(X\) que haga continua \(f\).
¡Mejor la más gruesa!

\(\T_X\defeq\{f^{-1}(V): V ∈ \T_Y\}\), ¿será topología?
\begin{enumerate}
  \item ¿\(∅ ∈ \T_X\)? \(f^{-1}(∅) = ∅ ∈ \T_X\). \\
        ¿\(X ∈ \T_X\)? \(f^{-1}(X) = X ∈ \T_X\). \\
  \item \(U_α ∈ \T_X\; ∀α ∈ \mathscr{A}\).
    Por demostrar:
    \(\bigcup_{α ∈ \mathscr{A}} U_α ∈ \T_X\).
    \begin{equation}
      U_α ∈ \T_X
        \quad⇒\quad U_α = f^{-1}(V_α) \text{ con } V_α ∈ \T_Y\quad ∀α ∈ \mathscr{A}
    \end{equation}
    \begin{equation}
      x ∈ \bigcup_{α ∈ \mathscr{A}} U_α
      \stackrel{?}{=}
      f^{-1}\Bigl(
        \underbrace{\bigcup_{α ∈ \mathscr{A}} V_α}_{∈ \T_Y}
      \Bigr)
    \end{equation}
    \begin{align}
      x ∈ \bigcup_{α ∈ \mathscr{A}} U_α\quad
        ⇔\quad& ∃α ∈ \mathscr{A}: x ∈ U_α = f^{-1}(V_α) \\
        ⇔\quad& f(x) ∈ V_{α_0} \\
        ⇔\quad& f(x) ∈ \bigcup_{α ∈ \mathscr{A}} V_α \\
        ⇔\quad& x ∈ f^{-1}(\bigcup_{α ∈ \mathscr{A}} V_α).
    \end{align}
  \item
    (Falta diagrama).
\end{enumerate}
