\section*{Lunes 30 de mayo}

\(X ≠ ∅\) conjunto, \(Y, \T_Y\) espacio topológico.

\(f\colon X\to Y\) pull-back de la topología \(\T_Y\) al conjunto \(X\):
\begin{align}
  \T_X      &\defeq \{f^{-1}(V): V ∈ \T\}, \\
  f^{-1}(V) &\defeq \{x ∈ X: f(x) ∈ V\} ⊆ X.
\end{align}
\(\T_X\) es la topología débil de \(X\).
\(f\) es automáticamente continua en \(\T_X\).

(Una función es o no continua
dependiendo de la topología en la que está definida).

Lo mismo pero con una familia de funciones:
\begin{equation}
  X \xrightarrow{f_α} (Y_α, \T_Y^α),\quad α ∈ \mathscr{A}
\end{equation}

...

Otrosí.
\(X ≠ ∅\), sea \(\rel\) una relación de equivalencia sobre \(X\).
Consideremos \(X/\rel \defeq\)
conjunto de todas las clases de \(\rel\)-equivalencia
en \(X = \{ {[x]}_{\rel}: x ∈ X\}\).

\begin{align}
  \label{eq:}
  π\colon X &\to Y = X/\rel \\
  x &\mapsto [x]_{\rel}.
\end{align}
Supongamos que \(X\) tenga una topología \(\T\).
Problema: definir una topología sobre \(Y\)
tal que \(π\) sea continua.
\begin{align}
  π^{-1}(y) &= \{ ξ ∈ X: ξ\rel x\} \quad\xrightarrow{π}\quad [x]_{\rel} = y \\
  π(\underbrace{U}_{∈ \T_X}) &⊂ X/\rel \\
  \mathfrak{M}_Y &= \{ π(U): U ∈ \T \} \\
  π(u) &= [u]_{\rel}.
\end{align}

(Falta diagrama).

\begin{equation}
  \mathscr{L}([0, 1], \R) = \{ f\colon [0, 1]\to\R:
  \text{\(f\) es medible e integrable} \}.
\end{equation}
\begin{equation}
  f = g \quad⇔\quad \int_0^1 \abs{f(x) - g(x)}\,dx = 0
\end{equation}
es relación de equivalencia \(→ \mathscr{L}(\ldots)/\rel = L^1(\ldots)\).

\((X, \T)\) espacio topológico.
Sea una familia \(\mathfrak{A}\) de conjuntos abiertos (i.e. \(U ∈ \T\)) de \(X\).
Se dice que \(\mathfrak{A}\) es un \emph{recubrimiento abierto de \(X\)}
ssi \(\bigcup_{U ∈ \mathfrak{A}} U = X\).

Se dice que \(X\) es \emph{compacto} ssi
todo recubrimiento abierto de \(X\)
posee un subrecubrimiento finito.

Se dice que \(K ⊆ X\) es compacto ssi \((K, \T_K)\) es compacto,
donde \(\T_K\) es la topología que \(K\) hereda de \(X\).

Teorema.
En \(\R^n\),
los conjuntos compactos son exactamente los conjuntos cerrados y acotados
(Heine, Borel, Lebesgue).
