\section*{Lunes 4 de julio---Linealidad y continuidad}

\(X\), \(Y\) espacios normados,
\(\norm{\,\mathord\cdot\,}\) denota la norma en cualquiera de los espacios;
el contexto indicará cuál.
(En \(\R^n\) y \(\C^n\) todas las normas
son prácticamente equivalentes,
pero en general no es así).

Sea \(T\colon X\to Y\) lineal.
Entonces las siguientes propiedades son equivalentes (¡tarea!):
\begin{enumerate}
\item \(T\) es continua en \(0 ∈ X\).
\item \(T\) es continua en un \(x ∈ X\) fijo.
\item \(T\) es continua sobre \(X\).
\end{enumerate}

Caso particular importante: \(Y = \K\).
Esto da origen a \(X^{*}\defeq L(X, \K)\)
(el conjunto de todas las aplicaciones lineales \(T\colon X\to\K\)).

\(X^{*}\) es el \emph{espacio dual algebraico} de \(X\),
y tiene estructura de \(\K\)-espacio vectorial en forma natural.
Más sobre esto más adelante.

\(S, T ∈ L(X, \K)\), \(α, β ∈ \K\). Entonces:
\begin{equation}
αS + βT:\quad
\bigl(αS + βT\bigr)(x)\defeq
\underbrace{
  \underbrace{α}_{∈\K}
  \underbrace{S(x)}_{∈\K} +
  \underbrace{β}_{∈\K}
  \underbrace{T(x)}_{∈\K}
}_{∈\K}
\qquad ∀x ∈ X
\end{equation}

\(X^{*} = \bigl(L(X, \K), +, \cdot\K\bigr)\) es un \(\K\)-espacio vectorial.

Ejemplo:
\(X\defeq L^2([0, 1], \R) = 
\bigl\{
  f\colon[0, 1]\to\R:
  f\text{ integrable} ∧ \int_0^1 \abs{f(x)}^2\,dx < 0
\bigr\}
\).
\begin{align}
φ_λ\colon X &\to\R \\
f &\mapsto φ_λ(x)\defeq \int_0^1 f(x) λ(x)\,dx,
\end{align}
donde \(λ\) es una función continua definida sobre \([0, 1]\).
Por ejemplo, si \(λ_k = x^k\), entonces
\(φ_{x^k} = \int_0^1 f(x) x^k\,dx\).
Diremos que \(λ(x)\) es una \emph{medida} sobre \([0, 1]\).

Notar que \(L^2([0, 1], \R\) es un espacio vectorial sobre \(\R\).
\begin{equation}
f, g ∈ L^2([0, 1], \R),\quad
α, β ∈ \R \quad⇒\quad
\bigl(αf + βg\bigr)(t)\defeq αf(t) + βg(t).
\end{equation}

(Los espacios de Hilbert son su propio dual).

Obs.
El espacio \(L^2([0, 1], \R\) es, en realidad, un \emph{espacio de Hilbert}
(adelantándonos un poco a los temas que tenemos que discutir).
El producto interno en este espacio viene dado por:
\begin{equation}
\label{eq:prod-interno-l2}
\inprod{f, g}\defeq
\int_0^1 f(x) g(x) λ(x)\,dx \quad ∈ \R\quad ∀f, g ∈ L^2([0, 1], \R).
\end{equation}
Habitualmente se exige que \(λ(x) ≥ 0\) y \(\int_0^1 λ(x)\,dx = 1\).

Que \eqref{eq:prod-interno-l2} esté bien definido
es una consecuencia de la \emph{desigualdad de Cauchy-Schwartz}:
\begin{equation*}
\left\lvert \int_0^1 f(x) g(x) λ(x)\,dx \right\rvert
≤
\left( \int_0^1 \abs{f(x)}^2 λ(x)\,dx \right)^{\!1/2}
\left( \int_0^1 \abs{g(x)}^2 λ(x)\,dx \right)^{\!1/2}
\end{equation*}
Ademas, \(\norm{f} \defeq \bigl(\int_0^1 \abs{f(x)}^2 λ(x)\,dx\bigr)^{1/2}\)
es una norma sobre \(L^2([0, 1], \R)\).

Otro caso importante es \(Y = X\).
\(\End(X)\defeq L(X, X)\) es el \emph{álgebra de los endomorfismos de \(X\)}.
\(f ∈ X, f\colon X\to X\) es un endomorfismo.

Los endomorfismos se pueden componer:
\(f\colon X\to X\),
\(g\colon X\to X\), entonces:
\begin{equation}
\bigl(f\circ g\bigr)(x) \defeq f\bigl(g(x)\bigr).
\end{equation}

En \(\End(X)\) tenemos tres operaciones:
\(\mathord{+}\),
\(\mathord{\cdot\K}\) y
\(\mathord{\circ}\).
\(\End(X)\) constituye una \(\K\)-álgebra.

Objetivo: definir bien una norma sobre \(L(X, Y)\).

Lema: \(X, Y\) espacios normados, \(T\colon X\to Y\) lineal.
Entonces:
\begin{equation}
\text{\(T\) es continua} \quad⇔\quad
∃ C ≥ 0 \;
∀ x ∈ X:\;
\norm{T(x)} ≤ C\norm{x}.
\end{equation}

Demostración:

\((\mathord{\Leftarrow})\)
¡Tarea! (como \(T\) es lineal, basta revisar en el origen).

\((\mathord{⇒})\)
(Por transcribir).

Consecuencia:
hay muchas (¡infinitas!) constantes \(C > 0\) como dice el lema.
\begin{equation}
\{C ≥ 0: \norm{T(x)} ≤ C\norm{x}\;∀x ∈ X\} ≠ ∅.
\end{equation}

Es del todo natural tomar el ínfimo de este conjunto,
y esta es una norma llamada \emph{norma como operador}
de la aplicación lineal y continua \(T\):
\begin{equation}
\norm{T}_{\text{op}} \defeq
\inf \{C ≥ 0: \norm{T(x)} ≤ C\norm{x}\;∀x ∈ X\} ≠ ∅.
\end{equation}
Demostraremos que \(\norm{T}\) también se puede escribir como:
\begin{equation}
\norm{T}_{\text{op}} =
\sup_{0 ≠ x ∈ X} \frac{\norm{T(x)}_Y}{\norm{x}_X} =
\sup_{\norm{x}_X = 1} \norm{T(x)}_Y.
\end{equation}


